\documentclass{article}

\usepackage{cite}

\title{JKind}
\author{Andrew Gacek, John Backes, Mike Whalen, Lucas Wagner}

\begin{document}
\maketitle

\section{Introduction}

\begin{itemize}
\item Lustre
\item Safety properties / monitors
\end{itemize}

\section{Functionality and Main Features}

\begin{itemize}
\item k-induction
\item Template invariant generation
\item PDR (+IA)
\item Interval Generalization
\item IVC
\item Advice
\item Smoothing
\end{itemize}

\section{Integration \& Applications}

\begin{itemize}
\item AGREE
  \begin{itemize}
  \item Infusion pump
  \item QFCS
  \item WBS
  \item ULB LOI
  \end{itemize}
\item SpeAR~\cite{fifarek2017nfm}
\item SIMPAL~\cite{wagner2017spin}
\item Internal MC/DC test case generation (public?)
\item Model-based fuzzing (public?)
\item Other speculative work
\end{itemize}

\section{Dicussion \& Related Work}

\begin{itemize}
\item Benchmarks: reference Kind 2 paper and say we are competitive on
  speed and strength but initial start up time due to JVM
\item Relationship to Kind 2~\cite{champion2016cav}
  \begin{itemize}
  \item Both based on PKind
  \item Independently developed from Kind 2, released a couple years earlier
  \item Multiplatform due to Java (Windows, Linux, Mac)
  \item JKind has focused on integration from the very beginning. Used
    in AGREE and internal MC/DC test case generation work from day
    one. JKind uses SMTInterpol by default so that it can be packaged
    into other applications and run everywhere with no setup required.
    We also provide JKindApi for things like UI, etc.
  \end{itemize}
\end{itemize}

\section{Experimental \& Future Work}

\begin{itemize}
\item Inductive datatypes
\item Arrays
\item Uninterpreted functions
\end{itemize}

\bibliography{main}{}
\bibliographystyle{plain}

\end{document}




% Local Variables:
% TeX-master: "main.tex"
% End:
